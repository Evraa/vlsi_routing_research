\subsection{Basics} %There's a space here, don't erase it.
\begin{itemize}
	\item Maze: it's a $[D,W,H]$ matrix that contains cells, used to simulate the grid.
    \item D: number of layers, W: width of the layer, H: height of the layer.
    \item Cells: each cell is a pin, a cell could represent a pin or an obstalce.
	\item Pin: the part where transistors gets connected to.
	\item Source: the starting point (pin) that needs to be connected to some targets.
	\item Tatgets: one or many point/s (pin/s) that need to be connected to the source in minimum cost.
	\item Obstacle: a block that wires can't go through.
	\item Vias: like a ladder to the upper or lower layer.
	\item Wire: what connects the pins with each others.
	\item Path: the route which the wire will take in order to connect the source with all of the targets.
	\item Cost: the length of the path, the longer the wire the larger the delay.
	\item Multi-layers: instead of having only one $2D$ grid, we have multiple grids, stacked vertically.
	\item Steiner Point: intermediate points that targets can be connected to.
\end{itemize}

\subsection{Assumptions} 
\begin{itemize}
    \item All wires are the same size.
    \item No geometric rules violations(ie. spacing).
    \item All pins are placed at the center of cells.
\end{itemize}
