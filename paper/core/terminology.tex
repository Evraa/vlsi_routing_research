\subsection{Basic Terms} %There's a space here, don't erase it.

Maze: it's a $[W x H]$ matrix that contains cells, used to simulate the grid.

Cells: each cell is a pin, a cell could represent a pin or an obstalce.

Pin: the part where transistors gets connected to.

Source: the starting point (pin) that needs to be connected to some targets.

Tatgets: one or many point/s (pin/s) that needs to be connected to the source in minimum cost.

Obstacle: a block that wires can't go through.

Vias: like a ladder to the upper or lower layer.

Wire: what connects the pins with each others.

Path: the route which the wire will take in order to connect the source with all of the targets.

Cost: the length of the path, the longer the wire the longer the delay.

Multi-layers: instead of having only one $2D$ grid, we have multiple grids, stacked vertically.

Steiner Point: intermediate points that targets can be connected to.


\subsection{Assumptions} 
All wires are the same size.

No geometric rules (ie. spacing) violations.

All pins are the center of cells.